
% Default to the notebook output style

    


% Inherit from the specified cell style.




    
\documentclass[11pt]{article}

    
    
    \usepackage[T1]{fontenc}
    % Nicer default font (+ math font) than Computer Modern for most use cases
    \usepackage{mathpazo}

    % Basic figure setup, for now with no caption control since it's done
    % automatically by Pandoc (which extracts ![](path) syntax from Markdown).
    \usepackage{graphicx}
    % We will generate all images so they have a width \maxwidth. This means
    % that they will get their normal width if they fit onto the page, but
    % are scaled down if they would overflow the margins.
    \makeatletter
    \def\maxwidth{\ifdim\Gin@nat@width>\linewidth\linewidth
    \else\Gin@nat@width\fi}
    \makeatother
    \let\Oldincludegraphics\includegraphics
    % Set max figure width to be 80% of text width, for now hardcoded.
    \renewcommand{\includegraphics}[1]{\Oldincludegraphics[width=.8\maxwidth]{#1}}
    % Ensure that by default, figures have no caption (until we provide a
    % proper Figure object with a Caption API and a way to capture that
    % in the conversion process - todo).
    \usepackage{caption}
    \DeclareCaptionLabelFormat{nolabel}{}
    \captionsetup{labelformat=nolabel}

    \usepackage{adjustbox} % Used to constrain images to a maximum size 
    \usepackage{xcolor} % Allow colors to be defined
    \usepackage{enumerate} % Needed for markdown enumerations to work
    \usepackage{geometry} % Used to adjust the document margins
    \usepackage{amsmath} % Equations
    \usepackage{amssymb} % Equations
    \usepackage{textcomp} % defines textquotesingle
    % Hack from http://tex.stackexchange.com/a/47451/13684:
    \AtBeginDocument{%
        \def\PYZsq{\textquotesingle}% Upright quotes in Pygmentized code
    }
    \usepackage{upquote} % Upright quotes for verbatim code
    \usepackage{eurosym} % defines \euro
    \usepackage[mathletters]{ucs} % Extended unicode (utf-8) support
    \usepackage[utf8x]{inputenc} % Allow utf-8 characters in the tex document
    \usepackage{fancyvrb} % verbatim replacement that allows latex
    \usepackage{grffile} % extends the file name processing of package graphics 
                         % to support a larger range 
    % The hyperref package gives us a pdf with properly built
    % internal navigation ('pdf bookmarks' for the table of contents,
    % internal cross-reference links, web links for URLs, etc.)
    \usepackage{hyperref}
    \usepackage{longtable} % longtable support required by pandoc >1.10
    \usepackage{booktabs}  % table support for pandoc > 1.12.2
    \usepackage[inline]{enumitem} % IRkernel/repr support (it uses the enumerate* environment)
    \usepackage[normalem]{ulem} % ulem is needed to support strikethroughs (\sout)
                                % normalem makes italics be italics, not underlines
    

    
    
    % Colors for the hyperref package
    \definecolor{urlcolor}{rgb}{0,.145,.698}
    \definecolor{linkcolor}{rgb}{.71,0.21,0.01}
    \definecolor{citecolor}{rgb}{.12,.54,.11}

    % ANSI colors
    \definecolor{ansi-black}{HTML}{3E424D}
    \definecolor{ansi-black-intense}{HTML}{282C36}
    \definecolor{ansi-red}{HTML}{E75C58}
    \definecolor{ansi-red-intense}{HTML}{B22B31}
    \definecolor{ansi-green}{HTML}{00A250}
    \definecolor{ansi-green-intense}{HTML}{007427}
    \definecolor{ansi-yellow}{HTML}{DDB62B}
    \definecolor{ansi-yellow-intense}{HTML}{B27D12}
    \definecolor{ansi-blue}{HTML}{208FFB}
    \definecolor{ansi-blue-intense}{HTML}{0065CA}
    \definecolor{ansi-magenta}{HTML}{D160C4}
    \definecolor{ansi-magenta-intense}{HTML}{A03196}
    \definecolor{ansi-cyan}{HTML}{60C6C8}
    \definecolor{ansi-cyan-intense}{HTML}{258F8F}
    \definecolor{ansi-white}{HTML}{C5C1B4}
    \definecolor{ansi-white-intense}{HTML}{A1A6B2}

    % commands and environments needed by pandoc snippets
    % extracted from the output of `pandoc -s`
    \providecommand{\tightlist}{%
      \setlength{\itemsep}{0pt}\setlength{\parskip}{0pt}}
    \DefineVerbatimEnvironment{Highlighting}{Verbatim}{commandchars=\\\{\}}
    % Add ',fontsize=\small' for more characters per line
    \newenvironment{Shaded}{}{}
    \newcommand{\KeywordTok}[1]{\textcolor[rgb]{0.00,0.44,0.13}{\textbf{{#1}}}}
    \newcommand{\DataTypeTok}[1]{\textcolor[rgb]{0.56,0.13,0.00}{{#1}}}
    \newcommand{\DecValTok}[1]{\textcolor[rgb]{0.25,0.63,0.44}{{#1}}}
    \newcommand{\BaseNTok}[1]{\textcolor[rgb]{0.25,0.63,0.44}{{#1}}}
    \newcommand{\FloatTok}[1]{\textcolor[rgb]{0.25,0.63,0.44}{{#1}}}
    \newcommand{\CharTok}[1]{\textcolor[rgb]{0.25,0.44,0.63}{{#1}}}
    \newcommand{\StringTok}[1]{\textcolor[rgb]{0.25,0.44,0.63}{{#1}}}
    \newcommand{\CommentTok}[1]{\textcolor[rgb]{0.38,0.63,0.69}{\textit{{#1}}}}
    \newcommand{\OtherTok}[1]{\textcolor[rgb]{0.00,0.44,0.13}{{#1}}}
    \newcommand{\AlertTok}[1]{\textcolor[rgb]{1.00,0.00,0.00}{\textbf{{#1}}}}
    \newcommand{\FunctionTok}[1]{\textcolor[rgb]{0.02,0.16,0.49}{{#1}}}
    \newcommand{\RegionMarkerTok}[1]{{#1}}
    \newcommand{\ErrorTok}[1]{\textcolor[rgb]{1.00,0.00,0.00}{\textbf{{#1}}}}
    \newcommand{\NormalTok}[1]{{#1}}
    
    % Additional commands for more recent versions of Pandoc
    \newcommand{\ConstantTok}[1]{\textcolor[rgb]{0.53,0.00,0.00}{{#1}}}
    \newcommand{\SpecialCharTok}[1]{\textcolor[rgb]{0.25,0.44,0.63}{{#1}}}
    \newcommand{\VerbatimStringTok}[1]{\textcolor[rgb]{0.25,0.44,0.63}{{#1}}}
    \newcommand{\SpecialStringTok}[1]{\textcolor[rgb]{0.73,0.40,0.53}{{#1}}}
    \newcommand{\ImportTok}[1]{{#1}}
    \newcommand{\DocumentationTok}[1]{\textcolor[rgb]{0.73,0.13,0.13}{\textit{{#1}}}}
    \newcommand{\AnnotationTok}[1]{\textcolor[rgb]{0.38,0.63,0.69}{\textbf{\textit{{#1}}}}}
    \newcommand{\CommentVarTok}[1]{\textcolor[rgb]{0.38,0.63,0.69}{\textbf{\textit{{#1}}}}}
    \newcommand{\VariableTok}[1]{\textcolor[rgb]{0.10,0.09,0.49}{{#1}}}
    \newcommand{\ControlFlowTok}[1]{\textcolor[rgb]{0.00,0.44,0.13}{\textbf{{#1}}}}
    \newcommand{\OperatorTok}[1]{\textcolor[rgb]{0.40,0.40,0.40}{{#1}}}
    \newcommand{\BuiltInTok}[1]{{#1}}
    \newcommand{\ExtensionTok}[1]{{#1}}
    \newcommand{\PreprocessorTok}[1]{\textcolor[rgb]{0.74,0.48,0.00}{{#1}}}
    \newcommand{\AttributeTok}[1]{\textcolor[rgb]{0.49,0.56,0.16}{{#1}}}
    \newcommand{\InformationTok}[1]{\textcolor[rgb]{0.38,0.63,0.69}{\textbf{\textit{{#1}}}}}
    \newcommand{\WarningTok}[1]{\textcolor[rgb]{0.38,0.63,0.69}{\textbf{\textit{{#1}}}}}
    
    
    % Define a nice break command that doesn't care if a line doesn't already
    % exist.
    \def\br{\hspace*{\fill} \\* }
    % Math Jax compatability definitions
    \def\gt{>}
    \def\lt{<}
    % Document parameters
    \title{BW\_141b\_hw3}
    
    
    

    % Pygments definitions
    
\makeatletter
\def\PY@reset{\let\PY@it=\relax \let\PY@bf=\relax%
    \let\PY@ul=\relax \let\PY@tc=\relax%
    \let\PY@bc=\relax \let\PY@ff=\relax}
\def\PY@tok#1{\csname PY@tok@#1\endcsname}
\def\PY@toks#1+{\ifx\relax#1\empty\else%
    \PY@tok{#1}\expandafter\PY@toks\fi}
\def\PY@do#1{\PY@bc{\PY@tc{\PY@ul{%
    \PY@it{\PY@bf{\PY@ff{#1}}}}}}}
\def\PY#1#2{\PY@reset\PY@toks#1+\relax+\PY@do{#2}}

\expandafter\def\csname PY@tok@w\endcsname{\def\PY@tc##1{\textcolor[rgb]{0.73,0.73,0.73}{##1}}}
\expandafter\def\csname PY@tok@c\endcsname{\let\PY@it=\textit\def\PY@tc##1{\textcolor[rgb]{0.25,0.50,0.50}{##1}}}
\expandafter\def\csname PY@tok@cp\endcsname{\def\PY@tc##1{\textcolor[rgb]{0.74,0.48,0.00}{##1}}}
\expandafter\def\csname PY@tok@k\endcsname{\let\PY@bf=\textbf\def\PY@tc##1{\textcolor[rgb]{0.00,0.50,0.00}{##1}}}
\expandafter\def\csname PY@tok@kp\endcsname{\def\PY@tc##1{\textcolor[rgb]{0.00,0.50,0.00}{##1}}}
\expandafter\def\csname PY@tok@kt\endcsname{\def\PY@tc##1{\textcolor[rgb]{0.69,0.00,0.25}{##1}}}
\expandafter\def\csname PY@tok@o\endcsname{\def\PY@tc##1{\textcolor[rgb]{0.40,0.40,0.40}{##1}}}
\expandafter\def\csname PY@tok@ow\endcsname{\let\PY@bf=\textbf\def\PY@tc##1{\textcolor[rgb]{0.67,0.13,1.00}{##1}}}
\expandafter\def\csname PY@tok@nb\endcsname{\def\PY@tc##1{\textcolor[rgb]{0.00,0.50,0.00}{##1}}}
\expandafter\def\csname PY@tok@nf\endcsname{\def\PY@tc##1{\textcolor[rgb]{0.00,0.00,1.00}{##1}}}
\expandafter\def\csname PY@tok@nc\endcsname{\let\PY@bf=\textbf\def\PY@tc##1{\textcolor[rgb]{0.00,0.00,1.00}{##1}}}
\expandafter\def\csname PY@tok@nn\endcsname{\let\PY@bf=\textbf\def\PY@tc##1{\textcolor[rgb]{0.00,0.00,1.00}{##1}}}
\expandafter\def\csname PY@tok@ne\endcsname{\let\PY@bf=\textbf\def\PY@tc##1{\textcolor[rgb]{0.82,0.25,0.23}{##1}}}
\expandafter\def\csname PY@tok@nv\endcsname{\def\PY@tc##1{\textcolor[rgb]{0.10,0.09,0.49}{##1}}}
\expandafter\def\csname PY@tok@no\endcsname{\def\PY@tc##1{\textcolor[rgb]{0.53,0.00,0.00}{##1}}}
\expandafter\def\csname PY@tok@nl\endcsname{\def\PY@tc##1{\textcolor[rgb]{0.63,0.63,0.00}{##1}}}
\expandafter\def\csname PY@tok@ni\endcsname{\let\PY@bf=\textbf\def\PY@tc##1{\textcolor[rgb]{0.60,0.60,0.60}{##1}}}
\expandafter\def\csname PY@tok@na\endcsname{\def\PY@tc##1{\textcolor[rgb]{0.49,0.56,0.16}{##1}}}
\expandafter\def\csname PY@tok@nt\endcsname{\let\PY@bf=\textbf\def\PY@tc##1{\textcolor[rgb]{0.00,0.50,0.00}{##1}}}
\expandafter\def\csname PY@tok@nd\endcsname{\def\PY@tc##1{\textcolor[rgb]{0.67,0.13,1.00}{##1}}}
\expandafter\def\csname PY@tok@s\endcsname{\def\PY@tc##1{\textcolor[rgb]{0.73,0.13,0.13}{##1}}}
\expandafter\def\csname PY@tok@sd\endcsname{\let\PY@it=\textit\def\PY@tc##1{\textcolor[rgb]{0.73,0.13,0.13}{##1}}}
\expandafter\def\csname PY@tok@si\endcsname{\let\PY@bf=\textbf\def\PY@tc##1{\textcolor[rgb]{0.73,0.40,0.53}{##1}}}
\expandafter\def\csname PY@tok@se\endcsname{\let\PY@bf=\textbf\def\PY@tc##1{\textcolor[rgb]{0.73,0.40,0.13}{##1}}}
\expandafter\def\csname PY@tok@sr\endcsname{\def\PY@tc##1{\textcolor[rgb]{0.73,0.40,0.53}{##1}}}
\expandafter\def\csname PY@tok@ss\endcsname{\def\PY@tc##1{\textcolor[rgb]{0.10,0.09,0.49}{##1}}}
\expandafter\def\csname PY@tok@sx\endcsname{\def\PY@tc##1{\textcolor[rgb]{0.00,0.50,0.00}{##1}}}
\expandafter\def\csname PY@tok@m\endcsname{\def\PY@tc##1{\textcolor[rgb]{0.40,0.40,0.40}{##1}}}
\expandafter\def\csname PY@tok@gh\endcsname{\let\PY@bf=\textbf\def\PY@tc##1{\textcolor[rgb]{0.00,0.00,0.50}{##1}}}
\expandafter\def\csname PY@tok@gu\endcsname{\let\PY@bf=\textbf\def\PY@tc##1{\textcolor[rgb]{0.50,0.00,0.50}{##1}}}
\expandafter\def\csname PY@tok@gd\endcsname{\def\PY@tc##1{\textcolor[rgb]{0.63,0.00,0.00}{##1}}}
\expandafter\def\csname PY@tok@gi\endcsname{\def\PY@tc##1{\textcolor[rgb]{0.00,0.63,0.00}{##1}}}
\expandafter\def\csname PY@tok@gr\endcsname{\def\PY@tc##1{\textcolor[rgb]{1.00,0.00,0.00}{##1}}}
\expandafter\def\csname PY@tok@ge\endcsname{\let\PY@it=\textit}
\expandafter\def\csname PY@tok@gs\endcsname{\let\PY@bf=\textbf}
\expandafter\def\csname PY@tok@gp\endcsname{\let\PY@bf=\textbf\def\PY@tc##1{\textcolor[rgb]{0.00,0.00,0.50}{##1}}}
\expandafter\def\csname PY@tok@go\endcsname{\def\PY@tc##1{\textcolor[rgb]{0.53,0.53,0.53}{##1}}}
\expandafter\def\csname PY@tok@gt\endcsname{\def\PY@tc##1{\textcolor[rgb]{0.00,0.27,0.87}{##1}}}
\expandafter\def\csname PY@tok@err\endcsname{\def\PY@bc##1{\setlength{\fboxsep}{0pt}\fcolorbox[rgb]{1.00,0.00,0.00}{1,1,1}{\strut ##1}}}
\expandafter\def\csname PY@tok@kc\endcsname{\let\PY@bf=\textbf\def\PY@tc##1{\textcolor[rgb]{0.00,0.50,0.00}{##1}}}
\expandafter\def\csname PY@tok@kd\endcsname{\let\PY@bf=\textbf\def\PY@tc##1{\textcolor[rgb]{0.00,0.50,0.00}{##1}}}
\expandafter\def\csname PY@tok@kn\endcsname{\let\PY@bf=\textbf\def\PY@tc##1{\textcolor[rgb]{0.00,0.50,0.00}{##1}}}
\expandafter\def\csname PY@tok@kr\endcsname{\let\PY@bf=\textbf\def\PY@tc##1{\textcolor[rgb]{0.00,0.50,0.00}{##1}}}
\expandafter\def\csname PY@tok@bp\endcsname{\def\PY@tc##1{\textcolor[rgb]{0.00,0.50,0.00}{##1}}}
\expandafter\def\csname PY@tok@fm\endcsname{\def\PY@tc##1{\textcolor[rgb]{0.00,0.00,1.00}{##1}}}
\expandafter\def\csname PY@tok@vc\endcsname{\def\PY@tc##1{\textcolor[rgb]{0.10,0.09,0.49}{##1}}}
\expandafter\def\csname PY@tok@vg\endcsname{\def\PY@tc##1{\textcolor[rgb]{0.10,0.09,0.49}{##1}}}
\expandafter\def\csname PY@tok@vi\endcsname{\def\PY@tc##1{\textcolor[rgb]{0.10,0.09,0.49}{##1}}}
\expandafter\def\csname PY@tok@vm\endcsname{\def\PY@tc##1{\textcolor[rgb]{0.10,0.09,0.49}{##1}}}
\expandafter\def\csname PY@tok@sa\endcsname{\def\PY@tc##1{\textcolor[rgb]{0.73,0.13,0.13}{##1}}}
\expandafter\def\csname PY@tok@sb\endcsname{\def\PY@tc##1{\textcolor[rgb]{0.73,0.13,0.13}{##1}}}
\expandafter\def\csname PY@tok@sc\endcsname{\def\PY@tc##1{\textcolor[rgb]{0.73,0.13,0.13}{##1}}}
\expandafter\def\csname PY@tok@dl\endcsname{\def\PY@tc##1{\textcolor[rgb]{0.73,0.13,0.13}{##1}}}
\expandafter\def\csname PY@tok@s2\endcsname{\def\PY@tc##1{\textcolor[rgb]{0.73,0.13,0.13}{##1}}}
\expandafter\def\csname PY@tok@sh\endcsname{\def\PY@tc##1{\textcolor[rgb]{0.73,0.13,0.13}{##1}}}
\expandafter\def\csname PY@tok@s1\endcsname{\def\PY@tc##1{\textcolor[rgb]{0.73,0.13,0.13}{##1}}}
\expandafter\def\csname PY@tok@mb\endcsname{\def\PY@tc##1{\textcolor[rgb]{0.40,0.40,0.40}{##1}}}
\expandafter\def\csname PY@tok@mf\endcsname{\def\PY@tc##1{\textcolor[rgb]{0.40,0.40,0.40}{##1}}}
\expandafter\def\csname PY@tok@mh\endcsname{\def\PY@tc##1{\textcolor[rgb]{0.40,0.40,0.40}{##1}}}
\expandafter\def\csname PY@tok@mi\endcsname{\def\PY@tc##1{\textcolor[rgb]{0.40,0.40,0.40}{##1}}}
\expandafter\def\csname PY@tok@il\endcsname{\def\PY@tc##1{\textcolor[rgb]{0.40,0.40,0.40}{##1}}}
\expandafter\def\csname PY@tok@mo\endcsname{\def\PY@tc##1{\textcolor[rgb]{0.40,0.40,0.40}{##1}}}
\expandafter\def\csname PY@tok@ch\endcsname{\let\PY@it=\textit\def\PY@tc##1{\textcolor[rgb]{0.25,0.50,0.50}{##1}}}
\expandafter\def\csname PY@tok@cm\endcsname{\let\PY@it=\textit\def\PY@tc##1{\textcolor[rgb]{0.25,0.50,0.50}{##1}}}
\expandafter\def\csname PY@tok@cpf\endcsname{\let\PY@it=\textit\def\PY@tc##1{\textcolor[rgb]{0.25,0.50,0.50}{##1}}}
\expandafter\def\csname PY@tok@c1\endcsname{\let\PY@it=\textit\def\PY@tc##1{\textcolor[rgb]{0.25,0.50,0.50}{##1}}}
\expandafter\def\csname PY@tok@cs\endcsname{\let\PY@it=\textit\def\PY@tc##1{\textcolor[rgb]{0.25,0.50,0.50}{##1}}}

\def\PYZbs{\char`\\}
\def\PYZus{\char`\_}
\def\PYZob{\char`\{}
\def\PYZcb{\char`\}}
\def\PYZca{\char`\^}
\def\PYZam{\char`\&}
\def\PYZlt{\char`\<}
\def\PYZgt{\char`\>}
\def\PYZsh{\char`\#}
\def\PYZpc{\char`\%}
\def\PYZdl{\char`\$}
\def\PYZhy{\char`\-}
\def\PYZsq{\char`\'}
\def\PYZdq{\char`\"}
\def\PYZti{\char`\~}
% for compatibility with earlier versions
\def\PYZat{@}
\def\PYZlb{[}
\def\PYZrb{]}
\makeatother


    % Exact colors from NB
    \definecolor{incolor}{rgb}{0.0, 0.0, 0.5}
    \definecolor{outcolor}{rgb}{0.545, 0.0, 0.0}



    
    % Prevent overflowing lines due to hard-to-break entities
    \sloppy 
    % Setup hyperref package
    \hypersetup{
      breaklinks=true,  % so long urls are correctly broken across lines
      colorlinks=true,
      urlcolor=urlcolor,
      linkcolor=linkcolor,
      citecolor=citecolor,
      }
    % Slightly bigger margins than the latex defaults
    
    \geometry{verbose,tmargin=1in,bmargin=1in,lmargin=1in,rmargin=1in}
    
    

    \begin{document}
    
    
    \maketitle
    
    

    
    \section{STA 141B: Homework 3}\label{sta-141b-homework-3}

Fall 2018

    \subsection{Information}\label{information}

After the colons (in the same line) please write just your first name,
last name, and the 9 digit student ID number below.

First Name: Bailey

Last Name: Wang

Student ID: 914955801

    \subsection{Instructions}\label{instructions}

\subsubsection{New item: Please print your answer notebook to pdf (make
sure that it is not too many pages, \textgreater{} 10, due to long
output) and submit as the homework solution with your zip
file.}\label{new-item-please-print-your-answer-notebook-to-pdf-make-sure-that-it-is-not-too-many-pages-10-due-to-long-output-and-submit-as-the-homework-solution-with-your-zip-file.}

We use a script that extracts your answers by looking for cells in
between the cells containing the exercise statements. So you

\begin{itemize}
\tightlist
\item
  MUST add cells in between the exercise statements and add answers
  within them and
\item
  MUST NOT modify the existing cells, particularly not the problem
  statement
\end{itemize}

To make markdown, please switch the cell type to markdown (from code) -
you can hit 'm' when you are in command mode - and use the markdown
language. For a brief tutorial see:
https://daringfireball.net/projects/markdown/syntax

    \subsubsection{Introduction}\label{introduction}

The US Department of Agriculture publishes price estimates for fruits
and vegetables
\href{https://www.ers.usda.gov/data-products/fruit-and-vegetable-prices/fruit-and-vegetable-prices/}{online}.
The most recent estimates are based on a 2013 survey of US retail
stores.

The estimates are provided as a collection of MS Excel files, with one
file per fruit or vegetable. The \texttt{hw3\_data.zip} file contains
the fruit and vegetable files in the directories \texttt{fruit} and
\texttt{vegetables}, respectively.

    \textbf{Exercise 1.} Use pandas to extract the "Fresh" row(s) from the
fruit Excel files. Combine the data into a single data frame. Your data
frame should look something like this:

\begin{longtable}[]{@{}lllllll@{}}
\toprule
type & food & form & price\_per\_lb & yield & lb\_per\_cup &
price\_per\_cup\tabularnewline
\midrule
\endhead
fruit & watermelon & Fresh1 & 0.333412 & 0.52 & 0.330693 &
0.212033\tabularnewline
fruit & cantaloupe & Fresh1 & 0.535874 & 0.51 & 0.374786 &
0.3938\tabularnewline
vegetables & onions & Fresh1 & 1.03811 & 0.9 & 0.35274 &
0.406868\tabularnewline
... & & & & & &\tabularnewline
\bottomrule
\end{longtable}

It's okay if the rows and columns of your data frame are in a different
order. These modules are especially relevant:

\begin{itemize}
\tightlist
\item
  \href{https://docs.python.org/2/library/stdtypes.html\#string-methods}{\texttt{str}
  methods}
\item
  \href{https://docs.python.org/2/library/os.html}{\texttt{os}}
\item
  \href{https://docs.python.org/2/library/os.path.html}{\texttt{os.path}}
\item
  \href{http://pandas.pydata.org/pandas-docs/stable/}{pandas}:
  \texttt{read\_excel()}, \texttt{concat()}, \texttt{.fillna()},
  \texttt{.str}, plotting methods
\end{itemize}

Ask questions and search the documentation/web to find the functions you
need.

    \begin{Verbatim}[commandchars=\\\{\}]
{\color{incolor}In [{\color{incolor}1}]:} \PY{c+c1}{\PYZsh{}Collabarated with Tiffany Chen}
        
        \PY{k+kn}{import} \PY{n+nn}{os}
        \PY{k+kn}{import} \PY{n+nn}{pandas} \PY{k}{as} \PY{n+nn}{pd}
        \PY{k+kn}{import} \PY{n+nn}{matplotlib}\PY{n+nn}{.}\PY{n+nn}{pyplot} \PY{k}{as} \PY{n+nn}{plt}
        \PY{k+kn}{import} \PY{n+nn}{numpy} \PY{k}{as} \PY{n+nn}{np}
        
        \PY{k}{def} \PY{n+nf}{fruit\PYZus{}function}\PY{p}{(}\PY{n}{file\PYZus{}name}\PY{p}{,} \PY{n}{subfile\PYZus{}name}\PY{p}{)}\PY{p}{:}
            
            \PY{l+s+sd}{\PYZdq{}\PYZdq{}\PYZdq{}Takes in a file path and the subfile which is the food type into the function. }
        \PY{l+s+sd}{     The function cleans the data by concating the food types into a new dataframe. }
        \PY{l+s+sd}{     The function also subsets the data into \PYZdq{}fresh\PYZdq{} and \PYZdq{}fresh1\PYZdq{} food types.}
        \PY{l+s+sd}{    }
        \PY{l+s+sd}{    Args:}
        \PY{l+s+sd}{        file\PYZus{}name: str}
        \PY{l+s+sd}{        subfile\PYZus{}name: str}
        \PY{l+s+sd}{        }
        \PY{l+s+sd}{    Returns:}
        \PY{l+s+sd}{        fruit1: Pandas dataframe}
        \PY{l+s+sd}{    \PYZdq{}\PYZdq{}\PYZdq{}}
            
            \PY{n}{fruit\PYZus{}path} \PY{o}{=} \PY{n}{os}\PY{o}{.}\PY{n}{path}\PY{o}{.}\PY{n}{join}\PY{p}{(}\PY{n}{file\PYZus{}name}\PY{p}{,} \PY{n}{subfile\PYZus{}name}\PY{p}{)}
            \PY{n}{ls\PYZus{}fruit\PYZus{}file} \PY{o}{=} \PY{n}{os}\PY{o}{.}\PY{n}{listdir}\PY{p}{(}\PY{n}{fruit\PYZus{}path}\PY{p}{)}
            
            \PY{n}{fruit1} \PY{o}{=} \PY{n}{pd}\PY{o}{.}\PY{n}{DataFrame}\PY{p}{(}\PY{p}{)}
            
            \PY{k}{for} \PY{n}{file} \PY{o+ow}{in} \PY{n}{ls\PYZus{}fruit\PYZus{}file}\PY{p}{:}
                \PY{n}{fruit2} \PY{o}{=} \PY{n}{pd}\PY{o}{.}\PY{n}{read\PYZus{}excel}\PY{p}{(}\PY{n}{fruit\PYZus{}path} \PY{o}{+} \PY{l+s+s2}{\PYZdq{}}\PY{l+s+s2}{/}\PY{l+s+s2}{\PYZdq{}} \PY{o}{+} \PY{n}{file}\PY{p}{,} \PY{n}{header} \PY{o}{=} \PY{l+m+mi}{1}\PY{p}{)}
                \PY{n}{fruit2}\PY{p}{[}\PY{l+s+s2}{\PYZdq{}}\PY{l+s+s2}{food}\PY{l+s+s2}{\PYZdq{}}\PY{p}{]} \PY{o}{=} \PY{n}{file}\PY{o}{.}\PY{n}{split}\PY{p}{(}\PY{l+s+s2}{\PYZdq{}}\PY{l+s+s2}{.}\PY{l+s+s2}{\PYZdq{}}\PY{p}{)}\PY{p}{[}\PY{l+m+mi}{0}\PY{p}{]}
                \PY{n}{fruit2}\PY{p}{[}\PY{l+s+s2}{\PYZdq{}}\PY{l+s+s2}{type}\PY{l+s+s2}{\PYZdq{}}\PY{p}{]} \PY{o}{=} \PY{n}{subfile\PYZus{}name}
                \PY{n}{fruit1} \PY{o}{=} \PY{n}{pd}\PY{o}{.}\PY{n}{concat}\PY{p}{(}\PY{p}{[}\PY{n}{fruit2}\PY{p}{,} \PY{n}{fruit1}\PY{p}{]}\PY{p}{,}\PY{n}{sort} \PY{o}{=} \PY{k+kc}{True}\PY{p}{)}
                
            \PY{n}{fruit1} \PY{o}{=} \PY{n}{fruit1}\PY{p}{[}\PY{n}{fruit1}\PY{p}{[}\PY{l+s+s2}{\PYZdq{}}\PY{l+s+s2}{Form}\PY{l+s+s2}{\PYZdq{}}\PY{p}{]}\PY{o}{.}\PY{n}{str}\PY{o}{.}\PY{n}{contains}\PY{p}{(}\PY{l+s+s2}{\PYZdq{}}\PY{l+s+s2}{Fresh}\PY{l+s+s2}{\PYZdq{}}\PY{p}{)}\PY{o}{.}\PY{n}{fillna}\PY{p}{(}\PY{k+kc}{False}\PY{p}{)}\PY{p}{]}
            
            \PY{n}{fruit1} \PY{o}{=} \PY{n}{fruit1}\PY{o}{.}\PY{n}{drop}\PY{p}{(}\PY{n}{columns} \PY{o}{=} \PY{p}{[}\PY{l+s+s2}{\PYZdq{}}\PY{l+s+s2}{Unnamed: 2}\PY{l+s+s2}{\PYZdq{}}\PY{p}{,} 
                                            \PY{l+s+s2}{\PYZdq{}}\PY{l+s+s2}{Unnamed: 5}\PY{l+s+s2}{\PYZdq{}}\PY{p}{,} 
                                            \PY{l+s+s2}{\PYZdq{}}\PY{l+s+s2}{Unnamed: 7}\PY{l+s+s2}{\PYZdq{}}\PY{p}{,} 
                                            \PY{l+s+s2}{\PYZdq{}}\PY{l+s+s2}{Unnamed: 8}\PY{l+s+s2}{\PYZdq{}}\PY{p}{]}\PY{p}{)}
            \PY{n}{fruit1} \PY{o}{=} \PY{n}{fruit1}\PY{o}{.}\PY{n}{rename}\PY{p}{(}\PY{n}{columns} \PY{o}{=} \PY{p}{\PYZob{}}\PY{l+s+s2}{\PYZdq{}}\PY{l+s+s2}{Average price}\PY{l+s+s2}{\PYZdq{}}\PY{p}{:} \PY{l+s+s2}{\PYZdq{}}\PY{l+s+s2}{price\PYZus{}per\PYZus{}cup}\PY{l+s+s2}{\PYZdq{}}\PY{p}{,} 
                                              \PY{l+s+s2}{\PYZdq{}}\PY{l+s+s2}{Average retail price }\PY{l+s+s2}{\PYZdq{}}\PY{p}{:} \PY{l+s+s2}{\PYZdq{}}\PY{l+s+s2}{price\PYZus{}per\PYZus{}lb}\PY{l+s+s2}{\PYZdq{}}\PY{p}{,}
                                              \PY{l+s+s2}{\PYZdq{}}\PY{l+s+s2}{Preparation}\PY{l+s+s2}{\PYZdq{}}\PY{p}{:} \PY{l+s+s2}{\PYZdq{}}\PY{l+s+s2}{yield}\PY{l+s+s2}{\PYZdq{}}\PY{p}{,} 
                                              \PY{l+s+s2}{\PYZdq{}}\PY{l+s+s2}{Size of a }\PY{l+s+s2}{\PYZdq{}}\PY{p}{:} \PY{l+s+s2}{\PYZdq{}}\PY{l+s+s2}{lb\PYZus{}per\PYZus{}cup}\PY{l+s+s2}{\PYZdq{}}\PY{p}{,}
                                              \PY{l+s+s2}{\PYZdq{}}\PY{l+s+s2}{Form}\PY{l+s+s2}{\PYZdq{}}\PY{p}{:} \PY{l+s+s2}{\PYZdq{}}\PY{l+s+s2}{form}\PY{l+s+s2}{\PYZdq{}}\PY{p}{\PYZcb{}}\PY{p}{)}
            \PY{n}{fruit1} \PY{o}{=} \PY{n}{fruit1}\PY{p}{[}\PY{p}{[}\PY{l+s+s2}{\PYZdq{}}\PY{l+s+s2}{type}\PY{l+s+s2}{\PYZdq{}}\PY{p}{,} \PY{l+s+s2}{\PYZdq{}}\PY{l+s+s2}{food}\PY{l+s+s2}{\PYZdq{}}\PY{p}{,} \PY{l+s+s2}{\PYZdq{}}\PY{l+s+s2}{form}\PY{l+s+s2}{\PYZdq{}}\PY{p}{,} 
                             \PY{l+s+s2}{\PYZdq{}}\PY{l+s+s2}{price\PYZus{}per\PYZus{}lb}\PY{l+s+s2}{\PYZdq{}}\PY{p}{,} 
                             \PY{l+s+s2}{\PYZdq{}}\PY{l+s+s2}{yield}\PY{l+s+s2}{\PYZdq{}}\PY{p}{,} 
                             \PY{l+s+s2}{\PYZdq{}}\PY{l+s+s2}{lb\PYZus{}per\PYZus{}cup}\PY{l+s+s2}{\PYZdq{}}\PY{p}{,} 
                             \PY{l+s+s2}{\PYZdq{}}\PY{l+s+s2}{price\PYZus{}per\PYZus{}cup}\PY{l+s+s2}{\PYZdq{}}\PY{p}{]}\PY{p}{]}
            \PY{n}{fruit1}\PY{o}{=} \PY{n}{fruit1}\PY{o}{.}\PY{n}{reset\PYZus{}index}\PY{p}{(}\PY{n}{drop}\PY{o}{=}\PY{k+kc}{True}\PY{p}{)}
            \PY{k}{return}\PY{p}{(}\PY{n}{fruit1}\PY{p}{)}
\end{Verbatim}


    \begin{Verbatim}[commandchars=\\\{\}]
{\color{incolor}In [{\color{incolor}2}]:} \PY{n}{fruit3} \PY{o}{=} \PY{n}{fruit\PYZus{}function}\PY{p}{(}\PY{l+s+s2}{\PYZdq{}}\PY{l+s+s2}{hw3\PYZus{}data/}\PY{l+s+s2}{\PYZdq{}}\PY{p}{,} \PY{l+s+s2}{\PYZdq{}}\PY{l+s+s2}{fruit}\PY{l+s+s2}{\PYZdq{}}\PY{p}{)} \PY{c+c1}{\PYZsh{}call for fruits}
        \PY{n}{fruit3}\PY{o}{.}\PY{n}{head}\PY{p}{(}\PY{p}{)}
\end{Verbatim}


\begin{Verbatim}[commandchars=\\\{\}]
{\color{outcolor}Out[{\color{outcolor}2}]:}     type          food    form price\_per\_lb yield lb\_per\_cup price\_per\_cup
        0  fruit    watermelon  Fresh1     0.333412  0.52   0.330693      0.212033
        1  fruit    tangerines  Fresh1      1.37796  0.74   0.407855      0.759471
        2  fruit  strawberries  Fresh1      2.35881  0.94    0.31967      0.802171
        3  fruit   raspberries  Fresh1      6.97581  0.96    0.31967       2.32287
        4  fruit   pomegranate  Fresh1      2.17359  0.56   0.341717       1.32634
\end{Verbatim}
            
    There are 24 rows in fruit data.

    \textbf{Exercise 2.} Reuse your code from exercise 1.1 to extract the
"Fresh" row(s) from the vegetable Excel files.

Does your code produce the correct prices for tomatoes? If not, why not?
Do any other files have the same problem as the tomatoes file?

You don't need to extract the prices for these problem files. However,
make sure the prices are extracted for files like asparagus that don't
have this problem.

    \begin{Verbatim}[commandchars=\\\{\}]
{\color{incolor}In [{\color{incolor}3}]:} \PY{c+c1}{\PYZsh{}Collabarated with Jared Yu}
        
        \PY{n}{vegetable1} \PY{o}{=}\PY{n}{fruit\PYZus{}function}\PY{p}{(}\PY{l+s+s2}{\PYZdq{}}\PY{l+s+s2}{hw3\PYZus{}data/}\PY{l+s+s2}{\PYZdq{}}\PY{p}{,} \PY{l+s+s2}{\PYZdq{}}\PY{l+s+s2}{vegetables}\PY{l+s+s2}{\PYZdq{}}\PY{p}{)} \PY{c+c1}{\PYZsh{}call for vegetables }
        
        \PY{n}{vegetable1}\PY{o}{.}\PY{n}{loc}\PY{p}{[}\PY{l+m+mi}{17}\PY{p}{,}\PY{l+s+s1}{\PYZsq{}}\PY{l+s+s1}{food}\PY{l+s+s1}{\PYZsq{}}\PY{p}{]}\PY{o}{=}\PY{l+s+s1}{\PYZsq{}}\PY{l+s+s1}{green\PYZus{}cabbage}\PY{l+s+s1}{\PYZsq{}} \PY{c+c1}{\PYZsh{}rename food type}
        \PY{n}{vegetable1}\PY{o}{.}\PY{n}{loc}\PY{p}{[}\PY{l+m+mi}{18}\PY{p}{,}\PY{l+s+s1}{\PYZsq{}}\PY{l+s+s1}{food}\PY{l+s+s1}{\PYZsq{}}\PY{p}{]}\PY{o}{=}\PY{l+s+s1}{\PYZsq{}}\PY{l+s+s1}{red\PYZus{}cabbage}\PY{l+s+s1}{\PYZsq{}}
        \PY{n}{vegetable1}\PY{o}{.}\PY{n}{loc}\PY{p}{[}\PY{l+m+mi}{24}\PY{p}{,}\PY{l+s+s1}{\PYZsq{}}\PY{l+s+s1}{food}\PY{l+s+s1}{\PYZsq{}}\PY{p}{]}\PY{o}{=}\PY{l+s+s1}{\PYZsq{}}\PY{l+s+s1}{unpeeled\PYZus{}cucumber}\PY{l+s+s1}{\PYZsq{}}
        \PY{n}{vegetable1}\PY{o}{.}\PY{n}{loc}\PY{p}{[}\PY{l+m+mi}{25}\PY{p}{,}\PY{l+s+s1}{\PYZsq{}}\PY{l+s+s1}{food}\PY{l+s+s1}{\PYZsq{}}\PY{p}{]}\PY{o}{=}\PY{l+s+s1}{\PYZsq{}}\PY{l+s+s1}{peeled\PYZus{}cucumber}\PY{l+s+s1}{\PYZsq{}}
        
        \PY{n}{vegetable1}\PY{o}{.}\PY{n}{head}\PY{p}{(}\PY{p}{)}
\end{Verbatim}


\begin{Verbatim}[commandchars=\\\{\}]
{\color{outcolor}Out[{\color{outcolor}3}]:}          type            food    form price\_per\_lb     yield lb\_per\_cup  \textbackslash{}
        0  vegetables   turnip\_greens  Fresh1      2.47175      0.75    0.31967   
        1  vegetables        tomatoes   Fresh          NaN       NaN        NaN   
        2  vegetables  sweet\_potatoes  Fresh1     0.918897  0.811301   0.440925   
        3  vegetables   summer\_squash  Fresh1      1.63948    0.7695   0.396832   
        4  vegetables         spinach  Fresh1          NaN       NaN        NaN   
        
          price\_per\_cup  
        0       1.05353  
        1           NaN  
        2        0.4994  
        3       0.84548  
        4           NaN  
\end{Verbatim}
            
    There are 33 rows in the vegetable data.

The vegetables with N/A values are tomatoes, spinach, mushrooms,
lettuce\_romaine, celery, cauliflower, broccoli, and carrots. In total
there are 8 rows with errors.

    \textbf{Exercise 3.} Remove rows without a price from the vegetable data
frame and then combine the fruit and vegetable data frames. Make sure
all columns of numbers are numeric (not strings).

    \begin{Verbatim}[commandchars=\\\{\}]
{\color{incolor}In [{\color{incolor}4}]:} \PY{c+c1}{\PYZsh{}Collabarated with Jared Yu}
        
        \PY{n}{vegetable2} \PY{o}{=} \PY{n}{vegetable1}\PY{o}{.}\PY{n}{dropna}\PY{p}{(}\PY{p}{)} \PY{c+c1}{\PYZsh{}drops na values in the dataframe}
        \PY{n}{food1} \PY{o}{=} \PY{p}{[}\PY{n}{fruit3}\PY{p}{,} \PY{n}{vegetable2}\PY{p}{]}
        \PY{n}{food2} \PY{o}{=} \PY{p}{[}\PY{l+s+s2}{\PYZdq{}}\PY{l+s+s2}{price\PYZus{}per\PYZus{}cup}\PY{l+s+s2}{\PYZdq{}}\PY{p}{,} \PY{l+s+s2}{\PYZdq{}}\PY{l+s+s2}{price\PYZus{}per\PYZus{}lb}\PY{l+s+s2}{\PYZdq{}}\PY{p}{,} \PY{l+s+s2}{\PYZdq{}}\PY{l+s+s2}{yield}\PY{l+s+s2}{\PYZdq{}}\PY{p}{,} \PY{l+s+s2}{\PYZdq{}}\PY{l+s+s2}{lb\PYZus{}per\PYZus{}cup}\PY{l+s+s2}{\PYZdq{}}\PY{p}{]}
        \PY{n}{food3} \PY{o}{=} \PY{n}{pd}\PY{o}{.}\PY{n}{concat}\PY{p}{(}\PY{n}{food1}\PY{p}{)}
        \PY{k}{for} \PY{n}{col} \PY{o+ow}{in} \PY{n}{food2}\PY{p}{:}
            \PY{n}{food3}\PY{p}{[}\PY{n}{col}\PY{p}{]} \PY{o}{=} \PY{n}{food3}\PY{p}{[}\PY{n}{col}\PY{p}{]}\PY{o}{.}\PY{n}{astype}\PY{p}{(}\PY{n+nb}{float}\PY{p}{)} \PY{c+c1}{\PYZsh{}in the for loop, }
        \PY{n}{food3}\PY{o}{.}\PY{n}{head}\PY{p}{(}\PY{p}{)}                             \PY{c+c1}{\PYZsh{}converts the listed objects into floats}
\end{Verbatim}


\begin{Verbatim}[commandchars=\\\{\}]
{\color{outcolor}Out[{\color{outcolor}4}]:}     type          food    form  price\_per\_lb  yield  lb\_per\_cup  price\_per\_cup
        0  fruit    watermelon  Fresh1      0.333412   0.52    0.330693       0.212033
        1  fruit    tangerines  Fresh1      1.377962   0.74    0.407855       0.759471
        2  fruit  strawberries  Fresh1      2.358808   0.94    0.319670       0.802171
        3  fruit   raspberries  Fresh1      6.975811   0.96    0.319670       2.322874
        4  fruit   pomegranate  Fresh1      2.173590   0.56    0.341717       1.326342
\end{Verbatim}
            
    There are 25 vegetables after dropping the N/A values.

There are 49 rows for the table with fruits and vegetables.

    \textbf{Exercise 4.} Discuss the questions below (a paragraph each is
sufficient). Use plots to support your ideas.

\begin{itemize}
\tightlist
\item
  What kinds of fruits are the most expensive (per pound)? What kinds
  are the least expensive?
\item
  How do the price distributions compare for fruit and vegetables?
\item
  Which foods are the best value for the price?
\item
  What's something surprising about this data set?
\item
  Which foods do you expect to provide the best combination of price,
  yield, and nutrition? A future assignment may combine this data set
  with another so you can check your hypothesis.
\end{itemize}

    \begin{Verbatim}[commandchars=\\\{\}]
{\color{incolor}In [{\color{incolor}6}]:} \PY{n}{fruit4} \PY{o}{=} \PY{n}{food3}\PY{p}{[}\PY{n}{food3}\PY{p}{[}\PY{l+s+s2}{\PYZdq{}}\PY{l+s+s2}{type}\PY{l+s+s2}{\PYZdq{}}\PY{p}{]} \PY{o}{==} \PY{l+s+s2}{\PYZdq{}}\PY{l+s+s2}{fruit}\PY{l+s+s2}{\PYZdq{}}\PY{p}{]} \PY{c+c1}{\PYZsh{}creates subgroup for only fruits}
        
        \PY{n}{x\PYZus{}pos} \PY{o}{=} \PY{n}{np}\PY{o}{.}\PY{n}{arange}\PY{p}{(}\PY{n+nb}{len}\PY{p}{(}\PY{n}{fruit4}\PY{p}{)}\PY{p}{)} \PY{c+c1}{\PYZsh{}creates tick locations for all the foods}
        \PY{n}{plt}\PY{o}{.}\PY{n}{bar}\PY{p}{(}\PY{n}{fruit4}\PY{p}{[}\PY{l+s+s2}{\PYZdq{}}\PY{l+s+s2}{food}\PY{l+s+s2}{\PYZdq{}}\PY{p}{]}\PY{p}{,} \PY{n}{fruit4}\PY{p}{[}\PY{l+s+s2}{\PYZdq{}}\PY{l+s+s2}{price\PYZus{}per\PYZus{}lb}\PY{l+s+s2}{\PYZdq{}}\PY{p}{]}\PY{p}{,} 
                    \PY{n}{align}\PY{o}{=}\PY{l+s+s1}{\PYZsq{}}\PY{l+s+s1}{center}\PY{l+s+s1}{\PYZsq{}}\PY{p}{,}\PY{n}{alpha}\PY{o}{=}\PY{o}{.}\PY{l+m+mi}{5}\PY{p}{,} \PY{n}{color}\PY{o}{=}\PY{l+s+s1}{\PYZsq{}}\PY{l+s+s1}{red}\PY{l+s+s1}{\PYZsq{}}\PY{p}{)} \PY{c+c1}{\PYZsh{}creates the graph}
        \PY{n}{plt}\PY{o}{.}\PY{n}{xticks}\PY{p}{(}\PY{n}{x\PYZus{}pos}\PY{p}{,} \PY{n}{fruit4}\PY{p}{[}\PY{l+s+s2}{\PYZdq{}}\PY{l+s+s2}{food}\PY{l+s+s2}{\PYZdq{}}\PY{p}{]}\PY{p}{,} \PY{n}{rotation}\PY{o}{=}\PY{l+m+mi}{90}\PY{p}{,} \PY{n}{fontsize}\PY{o}{=}\PY{l+m+mi}{16}\PY{p}{)} \PY{c+c1}{\PYZsh{}writes the labels}
        \PY{n}{plt}\PY{o}{.}\PY{n}{xlabel}\PY{p}{(}\PY{l+s+s1}{\PYZsq{}}\PY{l+s+s1}{Fruits}\PY{l+s+s1}{\PYZsq{}}\PY{p}{,} \PY{n}{fontsize}\PY{o}{=}\PY{l+m+mi}{13}\PY{p}{)} \PY{c+c1}{\PYZsh{}creates xlabel}
        \PY{n}{plt}\PY{o}{.}\PY{n}{ylabel}\PY{p}{(}\PY{l+s+s1}{\PYZsq{}}\PY{l+s+s1}{Price per lb (\PYZdl{})}\PY{l+s+s1}{\PYZsq{}}\PY{p}{,} \PY{n}{fontsize}\PY{o}{=}\PY{l+m+mi}{16}\PY{p}{)} \PY{c+c1}{\PYZsh{}creates ylabel}
        \PY{n}{plt}\PY{o}{.}\PY{n}{title}\PY{p}{(}\PY{l+s+s1}{\PYZsq{}}\PY{l+s+s1}{Fruits vs Price per lb}\PY{l+s+s1}{\PYZsq{}}\PY{p}{,}\PY{n}{fontsize}\PY{o}{=}\PY{l+m+mi}{18}\PY{p}{)} \PY{c+c1}{\PYZsh{}creates title}
        \PY{n}{plt}\PY{o}{.}\PY{n}{rcParams}\PY{p}{[}\PY{l+s+s1}{\PYZsq{}}\PY{l+s+s1}{figure.figsize}\PY{l+s+s1}{\PYZsq{}}\PY{p}{]} \PY{o}{=} \PY{p}{[}\PY{l+m+mi}{20}\PY{p}{,} \PY{l+m+mi}{5}\PY{p}{]} \PY{c+c1}{\PYZsh{}changes graph size}
        
        \PY{n}{plt}\PY{o}{.}\PY{n}{show}\PY{p}{(}\PY{p}{)} \PY{c+c1}{\PYZsh{}Prints graph}
\end{Verbatim}


    \begin{center}
    \adjustimage{max size={0.9\linewidth}{0.9\paperheight}}{output_15_0.png}
    \end{center}
    { \hspace*{\fill} \\}
    
    Most Exepensive Fruits:

Raspberries 6.97 per pound, Blackberries 5.77 per pound, Blueberries
4.73 per pound

Least Exepensive Fruits:

Watermelon .33 per pound, Cantaloupe .53 per pound, Bananas .56 per
pound

Raspberries are the most expensive fruits. Interestly, the reason
raspberries are so expensive is due to the fact that it has high
proudction cost and low yield. Also it can only grow in certain areas.
Watermelon are the least expensive fruits. These fruits contain a lot of
mass therefore, it will take less take to harvest watermelons compared
to berries.

    \begin{Verbatim}[commandchars=\\\{\}]
{\color{incolor}In [{\color{incolor}8}]:} \PY{n}{plt}\PY{o}{.}\PY{n}{hist}\PY{p}{(}\PY{n}{food3}\PY{p}{[}\PY{n}{food3}\PY{p}{[}\PY{l+s+s2}{\PYZdq{}}\PY{l+s+s2}{type}\PY{l+s+s2}{\PYZdq{}}\PY{p}{]}\PY{o}{==}\PY{l+s+s2}{\PYZdq{}}\PY{l+s+s2}{fruit}\PY{l+s+s2}{\PYZdq{}}\PY{p}{]}\PY{p}{[}\PY{l+s+s2}{\PYZdq{}}\PY{l+s+s2}{price\PYZus{}per\PYZus{}lb}\PY{l+s+s2}{\PYZdq{}}\PY{p}{]}\PY{p}{,} \PY{c+c1}{\PYZsh{}Make histogram}
                 \PY{n}{bins}\PY{o}{=}\PY{l+m+mi}{20}\PY{p}{,} \PY{n}{color}\PY{o}{=}\PY{l+s+s1}{\PYZsq{}}\PY{l+s+s1}{red}\PY{l+s+s1}{\PYZsq{}}\PY{p}{,}\PY{n}{alpha}\PY{o}{=}\PY{o}{.}\PY{l+m+mi}{5}\PY{p}{,} \PY{n}{label}\PY{o}{=}\PY{l+s+s2}{\PYZdq{}}\PY{l+s+s2}{Fruit}\PY{l+s+s2}{\PYZdq{}}\PY{p}{)}  \PY{c+c1}{\PYZsh{}for fruit}
        \PY{n}{plt}\PY{o}{.}\PY{n}{hist}\PY{p}{(}\PY{n}{food3}\PY{p}{[}\PY{n}{food3}\PY{p}{[}\PY{l+s+s2}{\PYZdq{}}\PY{l+s+s2}{type}\PY{l+s+s2}{\PYZdq{}}\PY{p}{]}\PY{o}{==}\PY{l+s+s2}{\PYZdq{}}\PY{l+s+s2}{vegetables}\PY{l+s+s2}{\PYZdq{}}\PY{p}{]}\PY{p}{[}\PY{l+s+s2}{\PYZdq{}}\PY{l+s+s2}{price\PYZus{}per\PYZus{}lb}\PY{l+s+s2}{\PYZdq{}}\PY{p}{]}\PY{p}{,} \PY{c+c1}{\PYZsh{}Make histogram}
                 \PY{n}{bins}\PY{o}{=}\PY{l+m+mi}{20}\PY{p}{,} \PY{n}{color}\PY{o}{=}\PY{l+s+s1}{\PYZsq{}}\PY{l+s+s1}{blue}\PY{l+s+s1}{\PYZsq{}}\PY{p}{,}\PY{n}{alpha}\PY{o}{=}\PY{o}{.}\PY{l+m+mi}{7}\PY{p}{,} \PY{n}{label}\PY{o}{=}\PY{l+s+s2}{\PYZdq{}}\PY{l+s+s2}{Vegetables}\PY{l+s+s2}{\PYZdq{}}\PY{p}{)} \PY{c+c1}{\PYZsh{}for vegetables}
        \PY{n}{plt}\PY{o}{.}\PY{n}{xlabel}\PY{p}{(}\PY{l+s+s1}{\PYZsq{}}\PY{l+s+s1}{Price per lb}\PY{l+s+s1}{\PYZsq{}}\PY{p}{,} \PY{n}{fontsize}\PY{o}{=}\PY{l+m+mi}{16}\PY{p}{)} \PY{c+c1}{\PYZsh{}creates xlabel}
        \PY{n}{plt}\PY{o}{.}\PY{n}{ylabel}\PY{p}{(}\PY{l+s+s1}{\PYZsq{}}\PY{l+s+s1}{Counts}\PY{l+s+s1}{\PYZsq{}}\PY{p}{,} \PY{n}{fontsize}\PY{o}{=}\PY{l+m+mi}{18}\PY{p}{)} \PY{c+c1}{\PYZsh{}creates ylabel}
        \PY{n}{plt}\PY{o}{.}\PY{n}{title}\PY{p}{(}\PY{l+s+s1}{\PYZsq{}}\PY{l+s+s1}{Foods vs Price per lb Distribution}\PY{l+s+s1}{\PYZsq{}}\PY{p}{,} \PY{n}{fontsize}\PY{o}{=}\PY{l+m+mi}{18}\PY{p}{)} \PY{c+c1}{\PYZsh{}creates title}
        \PY{n}{plt}\PY{o}{.}\PY{n}{legend}\PY{p}{(}\PY{n}{fontsize}\PY{o}{=}\PY{l+m+mi}{15}\PY{p}{)} \PY{c+c1}{\PYZsh{}Create legend}
        \PY{n}{plt}\PY{o}{.}\PY{n}{rcParams}\PY{p}{[}\PY{l+s+s1}{\PYZsq{}}\PY{l+s+s1}{figure.figsize}\PY{l+s+s1}{\PYZsq{}}\PY{p}{]} \PY{o}{=} \PY{p}{[}\PY{l+m+mi}{10}\PY{p}{,} \PY{l+m+mi}{5}\PY{p}{]} \PY{c+c1}{\PYZsh{}changes graph size}
        
        \PY{n}{plt}\PY{o}{.}\PY{n}{show}\PY{p}{(}\PY{p}{)} \PY{c+c1}{\PYZsh{}prints graph}
\end{Verbatim}


    \begin{center}
    \adjustimage{max size={0.9\linewidth}{0.9\paperheight}}{output_17_0.png}
    \end{center}
    { \hspace*{\fill} \\}
    
    Most Exepensive Fruits:

Raspberries 6.97 per pound, Blackberries 5.77 per pound Blueberries 4.73
per pound

Least Expensive Fruits:

Watermelon .33 per pound, Cantaloupe .53 per pound, Bananas .56 per
pound

Most Expensive Vegetables:

Okra 3.21 per pound, asparagus 3.21 per pound, Brussel Sprouts 2.76 per
pound

Least Expensive Vegetables:

Potatoes .56 per pound, Sweet potatoes .91 per pound, Onions 1.03 per
pound

The vegatables with higher costs are not as expensive as the fruits with
higher costs. Okra has a high cost due to the production during warmer
seasons. Therefore, during winter seasons, the US imports okra from
outside thus the cost increase. Potatoes are the cheapest, since they
require very little work to produce. The mass of potatoes are also
varies between medium and large potatoes.

The fruit distribution (red) shows a skewed-right distribution, which
might have a few outliers. The vegetable distribution (blue) shows a
bimodel, and does not have any extreme outliers.

    \begin{Verbatim}[commandchars=\\\{\}]
{\color{incolor}In [{\color{incolor}10}]:} \PY{n}{food3}\PY{p}{[}\PY{l+s+s2}{\PYZdq{}}\PY{l+s+s2}{value}\PY{l+s+s2}{\PYZdq{}}\PY{p}{]}\PY{o}{=}\PY{n}{food3}\PY{p}{[}\PY{l+s+s2}{\PYZdq{}}\PY{l+s+s2}{yield}\PY{l+s+s2}{\PYZdq{}}\PY{p}{]}\PY{o}{/}\PY{n}{food3}\PY{p}{[}\PY{l+s+s2}{\PYZdq{}}\PY{l+s+s2}{price\PYZus{}per\PYZus{}lb}\PY{l+s+s2}{\PYZdq{}}\PY{p}{]} \PY{c+c1}{\PYZsh{}Create new variable for value }
                                                             \PY{c+c1}{\PYZsh{}using yield and price\PYZus{}per\PYZus{}lb}
         \PY{n}{x\PYZus{}pos} \PY{o}{=} \PY{n}{np}\PY{o}{.}\PY{n}{arange}\PY{p}{(}\PY{n+nb}{len}\PY{p}{(}\PY{n}{food3}\PY{p}{)}\PY{p}{)} \PY{c+c1}{\PYZsh{}creates tick locations for all the foods}
         \PY{n}{plt}\PY{o}{.}\PY{n}{bar}\PY{p}{(}\PY{n}{x\PYZus{}pos}\PY{p}{,} \PY{n}{food3}\PY{p}{[}\PY{l+s+s2}{\PYZdq{}}\PY{l+s+s2}{value}\PY{l+s+s2}{\PYZdq{}}\PY{p}{]}\PY{p}{,}\PY{n}{align}\PY{o}{=}\PY{l+s+s1}{\PYZsq{}}\PY{l+s+s1}{center}\PY{l+s+s1}{\PYZsq{}}\PY{p}{,}\PY{n}{alpha}\PY{o}{=}\PY{o}{.}\PY{l+m+mi}{5}\PY{p}{,} \PY{n}{color}\PY{o}{=}\PY{l+s+s1}{\PYZsq{}}\PY{l+s+s1}{red}\PY{l+s+s1}{\PYZsq{}}\PY{p}{)} \PY{c+c1}{\PYZsh{}creates the graph}
         \PY{n}{plt}\PY{o}{.}\PY{n}{xticks}\PY{p}{(}\PY{n}{x\PYZus{}pos}\PY{p}{,} \PY{n}{food3}\PY{p}{[}\PY{l+s+s2}{\PYZdq{}}\PY{l+s+s2}{food}\PY{l+s+s2}{\PYZdq{}}\PY{p}{]}\PY{p}{,} \PY{n}{rotation}\PY{o}{=}\PY{l+m+mi}{90}\PY{p}{,} \PY{n}{fontsize}\PY{o}{=}\PY{l+m+mi}{16}\PY{p}{)} \PY{c+c1}{\PYZsh{}writes the labels}
         \PY{n}{plt}\PY{o}{.}\PY{n}{xlabel}\PY{p}{(}\PY{l+s+s1}{\PYZsq{}}\PY{l+s+s1}{Foods}\PY{l+s+s1}{\PYZsq{}}\PY{p}{,} \PY{n}{fontsize}\PY{o}{=}\PY{l+m+mi}{18}\PY{p}{)} \PY{c+c1}{\PYZsh{}creates xlabel}
         \PY{n}{plt}\PY{o}{.}\PY{n}{ylabel}\PY{p}{(}\PY{l+s+s1}{\PYZsq{}}\PY{l+s+s1}{Value}\PY{l+s+s1}{\PYZsq{}}\PY{p}{,} \PY{n}{fontsize}\PY{o}{=}\PY{l+m+mi}{16}\PY{p}{)} \PY{c+c1}{\PYZsh{}creates ylabel}
         \PY{n}{plt}\PY{o}{.}\PY{n}{title}\PY{p}{(}\PY{l+s+s1}{\PYZsq{}}\PY{l+s+s1}{Foods vs Value}\PY{l+s+s1}{\PYZsq{}}\PY{p}{,} \PY{n}{fontsize}\PY{o}{=}\PY{l+m+mi}{18}\PY{p}{)} \PY{c+c1}{\PYZsh{}creates title}
         \PY{n}{plt}\PY{o}{.}\PY{n}{rcParams}\PY{p}{[}\PY{l+s+s1}{\PYZsq{}}\PY{l+s+s1}{figure.figsize}\PY{l+s+s1}{\PYZsq{}}\PY{p}{]} \PY{o}{=} \PY{p}{[}\PY{l+m+mi}{20}\PY{p}{,} \PY{l+m+mi}{5}\PY{p}{]} \PY{c+c1}{\PYZsh{}changes graph size}
         
         \PY{n}{plt}\PY{o}{.}\PY{n}{show}\PY{p}{(}\PY{p}{)} \PY{c+c1}{\PYZsh{}Prints graph}
\end{Verbatim}


    \begin{center}
    \adjustimage{max size={0.9\linewidth}{0.9\paperheight}}{output_19_0.png}
    \end{center}
    { \hspace*{\fill} \\}
    
    bananas have cheap cost at .56 and medium yield .64

Watermelon have cheap cost at .33 and medium yield .52

Potato have cheap cost at .56 and a high yield .81

Sweet potato have cheap cost at .91 and a high yield .81

The formula for value = yield / price\_per\_lb

The fruits and vegetables varies between the cost and yield. The fruits
with low cost have rather medium yields. Rather the vegatables have low
cost and high yield. The best combination between cost and yield would
be low cost and high yield. Therefore, watermelon and okra are
considered the best value for price against yield.

I find it interesting that many of the vegetables have medium to low
yields but the price isn't low. Similarly, a lot of berries are
extremely expensive compared to other fruits. The cheapest fruits have
low yield, while the cheapest vegetables have the highest yield.


    % Add a bibliography block to the postdoc
    
    
    
    \end{document}
